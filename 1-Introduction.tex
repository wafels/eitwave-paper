\section{Introduction}\label{sec:Intro}

Extreme ultraviolet (EUV) waves are large-scale propagating
disturbances observed in the solar corona. These waves were discovered
through observations made by SOHO/EIT \citep{1997SoPh..175..571M,
  1998GeoRL..25.2465T, 1999ApJ...517L.151T}, and were hence initally
dubbed `EIT' waves. Since those first observations, over two hundred
papers discussing their properties, causes and physics have been
published. EUV waves appear to be strongly associated with CME
activity \cite{2002ApJ...569.1009B}, and to a lesser extent with solar
flares \citep{2006ApJ...641L.153}. However, at a fundamental level, the
physical nature of EUV waves is not completely understood.

In interpreting thse phenomena, some studies present evidence
supporting a magnetohydrodynamic (MHD) wave interpretation
\citep{1998GeoRL..25.2465T, 1999ApJ...517L.151T,2000ApJ...543L..89W,
  2001JGR...10625089W, 2002ApJ...574..440O, 2010ApJ...713.1008S},
others argue for what \cite{2012SoPh..281..187P} call a “pseudo-wave”
due to either the evolving manifestations of a CME
\citep{1999SoPh..190..107D, 2000ApJ...545..512D, 2008SoPh..247..123D,
  2011ApJ...738..167S} or transient localized brightenings
\citep{2007ApJ...656L.101A, 2007AN....328..760A}.  Some authors have
found evidence indicating that the complex brightenings associated
with EUV waves can be due to a combination of both MHD waves and
pseudo-waves \citep{2002ApJ...572L..99C, 2005ApJ...622.1202C,
  2004A&A...427..705Z, 2009ApJ...705..587C}.  It is clear from the
literature that the physical conditions that lead to the broad range
of observed wave propagation speeds \citep{2011A&A...532A.151W} and
amplitudes are poorly understood. EUV waves are also clearly
correlated with several other dynamic phenomena in addition to CMEs
and flares. For example, investigations such as
\cite{2000SoPh..193..161T}, \cite{2004A&A...427..705Z} and
\cite{2010ApJ...709..369P} have indicated that the development of
coronal dimmings may be closely linked to the development of EUV
waves.

The path of EUV waves have been observed to be modified by nearly all
major coronal features, including active regions
\citep{2000ApJ...543L..89W}, filaments \citep{2012ApJ...753...52L},
coronal holes \citep{2009ApJ...691L.123G}, streamers
\citep{2013ApJ...766...55K} and with varying degrees of transmission,
refraction, reflection and absorption reveal details about the wave
interaction with these features. However, the impulsive excitation,
and the range of interactions between the EUV waves and other coronal
structures are still unknown.

Some authors have also explored the potential of EUV waves in
diagnosing properties of the coronal medium that are otherwise hard to
measure, i.e., their use as tools to perform coronal seismology
\citep{1970PASJ...22..341U}. For example, if a fast MHD wave mode
interpretation is assumed, (see \cite{2011SSRv..158..365G}, for a
review of current interpretations of EUV waves), then the wave
propagation speed, coronal density and temperature can all be
estimated from observations, allowing the coronal magnetic field
strength to be derived \cite{2005LRSP....2....3N}.  This value can
also be used to test the accuracy of magnetic field extrapolation
codes \citep{2008ApJ...675.1637S} and other indirect measurements of
the coronal magnetic field strength \citep{2007Sci...317.1192T}.

%their fundamental nature is
%still not understood. In general, studies of this phenomenon can be
%assigned to at least one of these broad, non-exclusive categories: the
%physical nature and appearance of EUV waves, investigation of
%correlated phenomena, such as CMEs, flares, dimmings, and filament
%activity, probing the origin or driver of EUV waves, understanding the
%interaction with and impact on existing coronal features.

%In each of these categories, there have been major breakthroughs in
%the last several years, primarily due to the availability of
%high-cadence, multi-wavelength, multi-viewpoint observations from SDO,
%STEREO, Hinode, and other sources (for comprehensive reviews of recent
%results see \cite{2011SSRv..158..365G}; \cite{2011JASTP..73.1096Z};
%\cite{2011A&A...532A.151W}, \cite{2012SoPh..281..187P}).  Careful
%analysis has yielded a much improved understanding of the EUV wave
%phenomenon (e.g., Fig. 1), but it is clear that many outstanding
%questions remain.




% For example, \cite{2002ApJ...569.1009B}
%demonstrated that CMEs show a much greater association to EUV waves
%than do flares, while \cite{2006ApJ...641L.153} found that only
%eruptive flares were associated with EUV waves.  

Recent advances in solar instrumentation have allowed substantial
progress to be made over early SOHO/EIT observations. Data from
STEREO/EUVI provided a significant improvement in both spatial and
temporal resolution (e.g., \cite{2008ApJ...680L..81L,
  2008ApJ...681L.113V}.  Critically, with the launch of SDO in 2010,
highly detailed, multi-wavelength observations of EUV waves are now
possible (e.g. Fig. 2), illuminating the complex structure and
interactions of these waves (e.g., \cite{2012ApJ...753...52L}). With
these new data, studies of individual wave events (e.g.,
\cite{2011ApJ...741L..21L}) have augmented earlier kinematic studies
\citep{1999SoPh..190..467W, 2000ApJ...543L..89W}, improving the
description of the initiation and subsequent deceleration of EUV
waves.

However, a key limitation is that, in order to answer all of the
fundamental questions regarding the physical nature of EUV waves,
studying individual wave events or small samples is insufficient. To
make the next breakthrough in understanding this phenomenon requires
large-scale statistical studies with events robustly categorized by
their properties. Such studies are too time-intensive in scope to be
carried out manually. Hence, in this paper we present the Automated
Wave Analysis and Reduction in EUV (AWARE) algorithm, a new automated
EUV wave detection and characterization procedure applied to EUV image
data. Such a fully automated procedure is essential in order to unlock
the full potential of the large full-disk image datasets available
from SDO and STEREO, and enables the characterization of EUV waves in
large numbers. AWARE has been developed in the Python programming
language, and makes use of features provided by the SunPy data
analysis package \citep{mumford-proc-scipy-2013}. AWARE is also a
fully open-source, version controlled package, which is freely
available.

In Section 2 we discuss existing algorithms for the detection of solar
features, including EUV waves, and their current status. In Section 3
we discuss in detail the AWARE algorithm and pipeline. In Section 4 we
present a number of examples of successful detections with the
algorithm, and demonstrate its diagnostic and characterization
features.

 %AWARE will not only detect and characterize EUV waves in
%near real-time, but will also provide data products and event
%summaries available through the Solar Data Analysis Center.  The
%output of AWARE will substantially enhance the scientific community’s
%capabilities in EUV wave research, particularly in the following
%areas: 

%robustly determining whether different classes of EUV waves,
%with different physical characteristics, are supported by
%observations.  

%studying the relationship between EUV waves and
%associated phenomena, such as coronal dimmings and secondary
%eruptions.  

%using EUV waves to perform coronal seismology and better
%understand the properties of the solar corona.  

%the strength of
%correlations between various wave properties - such as velocity and
%amplitude - with the properties of the energetic events that produce
%these waves, for example CME eruption speed and flare magnitude.
