
%%%%%%%%%%%%%%%%%%%%%%%%%%%%%%%%%%%%%%%%%%%%%%%%%%%%%%%%%%%%%%%%%%%%%%%%%%%%%%%%
%%                                                                
%%      SWSC LaTeX class for Journal of Space Weather and Space Climate
%%      
%%                                      (c) Springer-Verlag HD
%%                                      revised by EDP Sciences
%%                                      further revised by J. Watermann 
%%
%%%%%%%%%%%%%%%%%%%%%%%%%%%%%%%%%%%%%%%%%%%%%%%%%%%%%%%%%%%%%%%%%%%%%%%%%%%%%%%%
%%
%%      This demonstration file was derived from aa.dem
%%  
%%      AA vers. 7.0, LaTeX class for Astronomy & Astrophysics
%%      demonstration file
%%                                                (c) Springer-Verlag HD
%%                                                revised by EDP Sciences
%%
%%%%%%%%%%%%%%%%%%%%%%%%%%%%%%%%%%%%%%%%%%%%%%%%%%%%%%%%%%%%%%%%%%%%%%%%%%%%%%%%
%%
%%      modified for Journal of Space Weather and Space Climate
%%      by Jurgen Watermann, Editorial Advisor to SWSC
%%
%%      01-04-2012
%%      02-04-2012 revision 1
%%      12-07-2012 revision 2
%%      06-12-2012 revision 3 
%%      01-01-2014 revision 4
%%      06-03-2014 revision 4.1
%%
%%%%%%%%%%%%%%%%%%%%%%%%%%%%%%%%%%%%%%%%%%%%%%%%%%%%%%%%%%%%%%%%%%%%%%%%%%%%%%%%
%%
%%      The two sub-figures referenced in this template are of eps and png type,
%%      respectively, in order to demonstrate the usepackages subfigure and
%%      epstopdf and thus create pdf-only output 
%%
%%      If you want to use TexLive or MikTex together with a bibtex bibliography 
%%      file you may run Latex2e from the command line 
%%          pdflatex -shell-escape swsc.tex
%%          bibtex swsc (do not include an extension such as .tex or .bib)
%%          pdflatex -shell-escape swsc.tex
%%          pdflatex -shell-escape swsc.tex
%%
%%      A double call to pdflatex after calling bibtex is necessary in order to
%%      set citations and references correctly and insure that foreward/backward  
%%      linkage (backref option) is properly applied
%%      If you use MikTex you may need to make a triple call to pdflatex
%%
%%      If you are using TexLive or MikTex but not a bibtex type of bibliography
%%      you may simply run Latex2e twice from the command line 
%%          pdflatex -shell-escape swsc.tex
%%          pdflatex -shell-escape swsc.tex
%%
%%%%%%%%%%%%%%%%%%%%%%%%%%%%%%%%%%%%%%%%%%%%%%%%%%%%%%%%%%%%%%%%%%%%%%%%%%%%%%%%
%%
%%   single column 12-point version for review
%%

%%  with traditional abstract
\documentclass[referee,a4paper,12pt,traditabstract]{swsc} 

%%  with structured abstract 
%\documentclass[referee,a4paper,12pt,structabstract]{swsc} 

\usepackage{graphicx}
\usepackage{txfonts}
\usepackage{subfigure}
\usepackage{epstopdf}
\usepackage{lineno}
\usepackage[authoryear,round]{natbib}
\usepackage[backref]{hyperref}
\usepackage{url}

%%    This version assumes using bibtex with the swsc bibliography style file
\bibliographystyle{swsc}

\hypersetup{colorlinks=true,citecolor=cyan,urlcolor=cyan,linkcolor=blue}

%%%%%%%%%%%%%%%%%%%%%%%%%%%%%%%%%%%%%%%%%%%%%%%%%%%%%%%%%%%%%%%%%%%%%%%%%%%%%%%%

\begin{document}

\begin{linenumbers}

\title{AWARE: An algorithm for the automated detection and characterization
  of EUV waves in the solar atmosphere}


   \subtitle{ }
   
   \titlerunning{An algorithm for the automated detection of EUV waves}

   \authorrunning{Ireland et al.}


%\input{authors}

\author{J. Ireland\inst{1,}\inst{2}\and
        A. R. Inglis\inst{2,}\inst{3}\and
        A. Y. Shih\inst{2}\and
        S. Christe\inst{2}\and
        S. Mumford\inst{4}\and
        L. A. Hayes\inst{5}\and
        B. J. Thompson\inst{2}
          }

\institute{ADNET Systems, Inc.\\
\email{\href{mailto:jack.ireland@nasa.gov}{jack.ireland@nasa.gov}}
\and
NASA Goddard Space Flight Center, MC 671.1, Greenbelt, MD 20771.
\and
Physics Department, The Catholic University of America,Washington, DC, 20064
\and
Solar Physics and Space Plasma Research Centre (SP$^{2}$RC), School of Mathematics and Statistics, The University of Sheffield, Hicks Building, Hounsfield Road, Sheffield S3 7RH, UK.
\and
School of Physics, Trinity College Dublin, Dublin 2, Ireland.
}


\abstract{Extreme ultraviolet (EUV) waves are large-scale propagating
disturbances observed in the solar corona, frequently associated with
coronal mass ejections and flares. Since their discovery, over two
hundred papers discussing their properties, causes and physical nature have
been published. However, despite this their fundamental properties and the physics of
their interactions with other solar phenomena are still not
understood. To further the understanding of EUV waves, we have constructed the Automated
Wave Analysis and REduction (AWARE) algorithm for the detection of EUV
waves over the full Sun. The AWARE algorithm is based on a novel image
processing approach to isolating the bright wavefront of the EUV as it
propagates across the corona.  AWARE detects the presence of a
wavefront, and measures the distance, velocity and acceleration of
that wavefront across the Sun.  The algorithm is applied to
simulations of EUV wave propagation and to observed data.  Suggestions
are also given for further refinements to the basic algorithm presented
here.}

\keywords{giant planet formation -- $\kappa$-mechanism -- stability of gas spheres}

\maketitle

\section{Introduction}\label{sec:Intro}

Extreme ultraviolet (EUV) waves are large-scale propagating
disturbances observed in the solar corona. These waves were discovered
through observations made by SOHO/EIT \citep{1997SoPh..175..571M,
  1998GeoRL..25.2465T, 1999ApJ...517L.151T}, and were hence initally
dubbed `EIT' waves. Since those first observations, over two hundred
papers discussing their properties, causes and physics have been
published. EUV waves appear to be strongly associated with CME
activity \cite{2002ApJ...569.1009B}, and to a lesser extent with solar
flares \citep{2006ApJ...641L.153}. However, at a fundamental level, the
physical nature of EUV waves is not completely understood.

In interpreting these phenomena, a variety of explanations have been put forward \citep[see][for a
review of current interpretations of EUV waves]{2011SSRv..158..365G}. Some studies present evidence
supporting a magnetohydrodynamic (MHD) wave interpretation
\citep{1998GeoRL..25.2465T, 1999ApJ...517L.151T,2000ApJ...543L..89W,
  2001JGR...10625089W, 2002ApJ...574..440O, 2010ApJ...713.1008S},
while others argue for what \cite{2012SoPh..281..187P} call a “pseudo-wave”
due to either the evolving manifestations of a CME
\citep{1999SoPh..190..107D, 2000ApJ...545..512D, 2008SoPh..247..123D,
  2011ApJ...738..167S} or transient localized brightenings
\citep{2007ApJ...656L.101A, 2007AN....328..760A}.  Some authors have
found evidence indicating that the complex brightenings associated
with EUV waves can be due to a combination of both MHD waves and
pseudo-waves \citep{2002ApJ...572L..99C, 2005ApJ...622.1202C,
  2004A&A...427..705Z, 2009ApJ...705..587C}. A unified explanation of this phenomenon is complicated by the broad range of observed wave propagation speeds \citep{2011A&A...532A.151W} and amplitudes of EUV waves. 

The path of EUV waves have been observed to be modified by nearly all
major coronal features, including active regions
\citep{2000ApJ...543L..89W}, filaments \citep{2012ApJ...753...52L},
coronal holes \citep{2009ApJ...691L.123G}, streamers
\citep{2013ApJ...766...55K} and with varying degrees of transmission,
refraction, reflection and absorption reveal details about the wave
interaction with these features. However, the impulsive excitation,
and the range of interactions between the EUV waves and other coronal
structures are still unknown. EUV waves are also clearly correlated with several other dynamic phenomena in addition to CMEs
and flares. For example, investigations such as
\cite{2000SoPh..193..161T}, \cite{2004A&A...427..705Z} and
\cite{2010ApJ...709..369P} have indicated that the development of
coronal dimmings may be closely linked to the development of EUV
waves.

Some authors have also explored the potential of EUV waves in
diagnosing properties of the coronal medium that are otherwise hard to
measure, i.e., their use as tools to perform coronal seismology
\citep{1970PASJ...22..341U}. For example, if a fast MHD wave mode
interpretation is assumed, then the wave
propagation speed, coronal density and temperature can all be
estimated from observations, allowing the coronal magnetic field
strength to be derived \citep{2005LRSP....2....3N}.  This value can
also be used to test the accuracy of magnetic field extrapolation
codes \citep{2008ApJ...675.1637S} and other indirect measurements of
the coronal magnetic field strength \citep{2007Sci...317.1192T}.


Recent advances in solar instrumentation have allowed substantial
progress to be made over early SOHO/EIT observations. Data from
STEREO/EUVI provided a significant improvement in both spatial and
temporal resolution (e.g., \cite{2008ApJ...680L..81L,
  2008ApJ...681L.113V}.  Critically, with the launch of SDO in 2010,
highly detailed, multi-wavelength observations of EUV waves are now
possible (e.g. Fig. 2), illuminating the complex structure and
interactions of these waves (e.g., \cite{2012ApJ...753...52L}). With
these new data, studies of individual wave events (e.g.,
\cite{2011ApJ...741L..21L}) have augmented earlier kinematic studies
\citep{1999SoPh..190..467W, 2000ApJ...543L..89W}, improving the
description of the initiation and subsequent deceleration of EUV
waves.

However, a key limitation is that, in order to answer all of the
fundamental questions regarding the physical nature of EUV waves,
studying individual wave events or small samples is insufficient. To
make the next breakthrough in understanding this phenomenon requires
large-scale statistical studies with events robustly categorized by
their properties. Such studies are too time-intensive in scope to be
carried out manually. Additionally it is challenging to produce statistically reproducable results of large samples with a manual approach. Automated feature detection algorithms have an advantage over human detections of features because they generate repeatable results for
the same input data, i.e. they enable reproducability. In addition,
their ability to examine large quantities of data faster than human
analysis is invaluable in the SDO era, which produces ~1 TB of solar
data each day. The solar physics community already makes use of the
Computer Aided CME Tracking (CACTus: \cite{2004A&A...425.1097R}) and
Solar Eruptive Event Detection System (SEEDS:
\cite{2008SoPh..248..485O}) CME catalogs, both of which are generated
from automated feature detection algorithms.

Hence, in this paper we present the Automated
Wave Analysis and Reduction in EUV (AWARE) algorithm, a new automated
EUV wave detection and characterization procedure applied to EUV image
data. Such a fully automated procedure is essential in order to unlock
the full potential of the large full-disk image datasets available
from SDO and STEREO, and enables the characterization of EUV waves in
large numbers. AWARE has been developed in the Python programming
language, and makes use of features provided by the SunPy data
analysis package \citep{mumford-proc-scipy-2013}. AWARE is also a
fully open-source, version controlled package, which is freely
available (see Section X).


In Section 2 we discuss existing algorithms for the detection of solar
features, including EUV waves, and their current status. In Section 3
we discuss in detail the AWARE algorithm and pipeline. In Section 4 we
present a number of examples of successful detections with the
algorithm, and demonstrate its diagnostic and characterization
features.


\section{Existing EUV wave detection algorithms}\label{sec:existing}

Automated feature detection algorithms have an advantage over human
detections of features because they generate repeatable results for
the same input data, i.e. they enable reproducability. In addition,
their ability to examine large quantities of data faster than human
analysis is invaluable in the SDO era, which produces ~1 TB of solar
data each day. The solar physics community already makes use of the
Computer Aided CME Tracking (CACTus: \cite{2004A&A...425.1097R}) and
Solar Eruptive Event Detection System (SEEDS:
\cite{2008SoPh..248..485O}) CME catalogs, both of which are generated
from automated feature detection algorithms.

When relying on such automated procedures, it is scientifically
valuable to have multiple independently designed and developed
automatic feature detection algorithms for the same type of feature as
they allow for cross-checking and verification of results.  By
comparing the final results, the operation of each feature detection
algorithm can be better understood. For example,
\cite{2013ApJ...768..162P} shows that eruption rates from both the
CACTus and SEEDS CME catalogs are systematically higher in 2003-2012
compared to 1997-2002, consistent with the weakness of the late cycle
23 polar fields.  Use of detections from two independently developed
automated CME detection algorithms greatly strengthens this result.

In terms of EUV waves, there are at least two automated EUV wave
detection methods currently published, the Novel EUV wave Machine
Observing algorithm, described by \cite{2005SoPh..228..265P} (see also
\cite{2012SoPh..276..479P}) and the Coronal Pulse Identification and
Tracking Algorithm described in \cite{2014SoPh..289.3279L}. NEMO was
originally designed for analysis of SOHO/EIT data, but has since been
modified to analyze STEREO/EUVI images. The original NEMO algorithm
\cite{2005SoPh..228..265P} consists of three components. These are 1)
source event detection, 2) recognition of eruptive dimmings, 3)
detection and analysis of EUV waves. The event detection component is
based on the higher-order moments of running difference (RD) images. A
RD image is simply the difference between two consecutive images. A
sharp change in the skewness or kurtosis of the distribution of RD
image values is a reliable signature that an impulsive event, such as
a flare or an EUV wave, has been observed in that image. Once a source
detection has occurred, the second and third components of NEMO are
triggered to identify eruptive dimmings and EUV waves respectively.
NEMO assumes that EUV waves propagate circularly from the originating
event. The wave front is found by integrating RD images in nested
annuli centered on the originating event.  Plotting these integrals as
a function of radial distance from the originating event shows a
leading intensity enhancement which is identified as the EUV wave.
Results from NEMO are available at \url{http://sidc.be/nemo/};
however, the implementation ceased operations in 2010 and so there are
no new EUV detections being provided to the community.
\cite{2012SoPh..276..479P} is concerned with advances to original NEMO
algorithm with respect to source event detection and eruptive
dimmings, and does not explicitly tackle EUV waves.

The second algorithm that has been developed is CorPITA
\citep{2014SoPh..289.3279L}. CorPITA uses percentage base-difference
(PBD) images as the foundation for detection (Fig. 2, top row).  A PBD
image is formed by taking the difference between a selected base image
and the current image, and then scaling that difference by the base
image, multiplied by 100.  CorPITA is triggered by the occurrence of a
flare.  In CorPITA PBD images, the base image is taken two minutes
before the flare start time. The flare position is used as the origin
of the EUV wave; great circles intersecting this origin are analyzed
to identify whether an EUV wave is present. The intensity profile
along the great circle is fitted for each time-step with a
multi-Gaussian function, based on the observation of
\cite{2006ApJ...645..757W} that cross-sections of EUV wave events have
this approximate form. This assumption allows the wave to be
characterized in terms of its position, velocity and width. Events and
data products generated by CorPITA are intended to be made available
within the HEK \citep{hek2012, 2012SoPh..275...79M}.  

In this context, AWARE provides a new, alternative approach for the
detection and characterization of EUV waves.  In the following
Sections, we describe in detail the imaging processing steps in AWARE,
and demonstrate its application to solar data.


\section{AWARE}\label{sec:aware}

\subsection{Image processing}
\label{img_proc}

As was noted in Sections 1 and 2, EUV waves are difficult to detect
since they are faint, extensive and propagate over a complex
background image (the solar corona). This realisation has driven past
attempts to enhance and detect EUV waves by making use of running
difference or base-difference images. However, these images, while
enhancing potential wavefronts, remain noisy and populated by other
extraneous features (see Figure \ref{rpdm_figure}). For AWARE, we
therefore choose a different approach, adopting a new, simple and very
promising strategy for segmenting an EUV wave wavefront from image
data. Instead of using traditional running- or base-difference image
processing, we make use of persistence imaging, as described by
\citet{2014AAS...22421838T}.

\subsection{Segmenting a wave from the image data}\label{sec:aware:segment}

A persistence image is found by calculating the persistence value of
the emission at each pixel at all locations and times.  The
persistence value at time $t$ of a time series $f(t)$ is simply the
maximum value reached by that pixel in the time range $0\rightarrow
t$.  If at later times the pixel value increases, the persistence
value increases accordingly. If the pixel value decreases however, the
persistence value remains unchanged. Hence, a set of persistence maps
constructed from an image sequence will indicate the brightest values
yet achieved in that sequence at each $t$.  The persistence transform
$P(t)$ of the time-series $f(t)$ is defined as
\begin{equation}
\label{eqn:persisttransform}
P(t) = \max_{t'=0}^{t'=t}f(t').
\end{equation}
Figure \ref{fig:persistence} shows the result of the pesistence
transform applied to some simulated data.  By obtaining sets of
persistence images from input solar EUV image data, and performing a
running difference operation on these images, rather than the original
input data, we are able to substantially enhance the appearance of
propagating waves. This is because running difference persistence
images (RDPIs) have two particularly desirable properties when
searching for EUV waves.  Firstly, only pixels that brighten over
previous values have a non-zero value in the running difference of
persistence images, while zero-value pixels correspond to areas that
did not increase in brightness. Hence, since an EUV wave brightens
neighboring pixels as it moves across the Sun, the RDPIs isolate those
brightening pixels.  In other words, the RDPIs isolate the leading
part of the wavefront that brightened new pixels.  Secondly, since
much of the corona does not vary significantly during an EUV wave,
RDPIs show very little residual coronal structure distant from the
EUV, greatly simplifying the resulting images.

\begin{figure}
\begin{center}
\includegraphics[width=16cm]{persistence_explanation.eps}
\caption{Example of the application of the persistence transform.  The
original data $f(t)$ is shown in blue, and its persistence transform
$P(t)$ is shown in green.}
\label{fig:persistence}
\end{center}
\end{figure}

\begin{figure}
\begin{center}
\includegraphics[width=16cm]{difference_comparison.eps}
\caption{Imaging processing techniques applied to three EUV waves,
  from 2011 October 1 (left column), 2011 February 13 (center column),
  and 2011 February 15 (right column) respectively. The top row shows
  percentage base difference (PBD) images of each event at a selected
  time, the image processing method used by the CorPITA algorithm
  \citep{2014SoPh..289.3279L}, while the second row shows the standard
  running difference (RD) images for these events (used in NEMO
  analysis, \citep{2005SoPh..228..265P}). In the bottom row, we show
  that the application of running difference persistence images (RDPI)
  as used by AWARE is able to extract the propagating EUV wave much
  more cleanly. Both Wave B and Wave C were analyzed in
  \cite{2014SoPh..289.3279L}; see their Fig. 7 and 8a respectively.}
\label{rpdm_figure}
\end{center}
\end{figure}

Figure \ref{rpdm_figure} illustrates the power of RDPIs for three
example EUV wave events from 2011 October 1 (Wave A), 2011 February 13
(Wave B), and 2011 February 15 (Wave C) respectively. For each event,
percentage base difference images (top row), running difference images
(center row) and RDPIs (bottom row) are shown. Waves B and C from
Figure \ref{rpdm_figure} were also analyzed by CorPITA in the
demonstration of that algorithm by \citet{2014SoPh..289.3279L}.  The
first row shows percentage base difference (PDB) images of each event,
the basic image type analyzed by the CorPITA algorithm.  The second
row shows running difference (RD) images, the basic image type
analyzed by the NEMO algorithm.  The third row shows RDPI images, the
basic image type analyzed by AWARE. Comparison with RDPIs shows that
in standard RD or PBD images the wavefront is more diffuse, and much
coronal structure not associated with the wavefront remains in the
image. RD and PBD images also have much denser noise compared to the
RDPIs of the same data; hence, separating the EUV wave from the noise
is substantially easier when using RDPIs.

Thus, the RPDI approach is the first step in the detection of EUV
waves with AWARE. Subsequent image processing steps allow us to refine
the detection of any propagating features. The major steps in the
AWARE detection and processing algorithm are described below, and are
demonstrated in Figure \ref{method_figure}.

\begin{enumerate}

\item Given a set of sequential solar EUV images (e.g. SDO/AIA or
  STEREO/EUVI data), data are summed in time and space to increase the
  signal to noise ratio of the wave against the background. Images may
  be summed in space as desired, for example an AIA image may be
  binned using $4\times4$ super-pixels to form $1024\times1024$ pixel
  images. In the time dimension, images may be summed as
  required. Typically, pairs or triplets of consecutive images are
  used.

\item The persistence transform is then applied to the resulting image
  set.  This creates a set of persistence images, showing the
  brightest values obtained in each pixel (see Equation
  \ref{eqn:persisttransform}) as a function of time.

\item Perform a running difference operation on the
  persistence images. Hence, only areas that increase in brightness
  from one time to the next remain (Figure \ref{method_figure}b).

\item 
The following steps are intended to isolate the wavefront at different
length-scales $r_{1}, r_{2}$\textellipsis, etc, from the noise in the
datacube.

\begin{enumerate}

\item Apply a noise reduction filter (Figure \ref{method_figure}c) to
  the data cube of images.  Our demonstration algorithm uses a simple
  two step process.  Firstly, all pixel values in the data cube above
  a certain threshold are set to zero. This filters out spikes in the
  data. Secondly, a median filter is applied.  This replaces every
  pixel in the image with the median value found in its neighborhood.
  The median filter used is a circle in the spatial dimensions with a
  given radius $r$.  The median filter also extends in the temporal
  dimension to the previous and next images in the data cube.  This
  extension in the temporal dimension improves the noise reduction
  since emission at recent times are compared.  The median filter is a
  commonly used and simple method of removing noise from an image
  \citep{2002dip..book.....G}.

\item Apply a morphological closing operation.  This operation helps
  close small ‘cracks’ in structures \citep{2002dip..book.....G}.  The
  structuring element used is the same as that used by the median
  filtering operation. The final image is shown in Figure
  \ref{method_figure}d.  The final result is a datacube of images that
  show the location of the wavefront as a function of time, given the
  median filter and closing operations performed at length-scale $r$.
\end{enumerate}

\item Add up the wave location datacubes over all the circle radii
  $r_{1}, r_{2}$\textellipsis, etc.  This is the final wave location
  datacube that defines the location of the wavefront as a function of
  time.

\end{enumerate}

The final product is a datacube of time-ordered series of images that
localize the bright wavefront of the EUV wave.  Figure
\ref{method_figure} shows each step in this procedure, applied to two
AIA 211 $\AA$ images during the EUV wave event of 2011 October 1 (Wave
A in Figure \ref{rpdm_figure}).  The result of the key step of
generating an RDPI (Steps 2 and 3) from the two input images is
illustrated in panel b). Steps 4 and 5 of the AWARE image processing
method (Figure \ref{method_figure}c, d) apply some basic noise
reduction and feature enhancement techniques with the intent to
further isolate the wave front in the data. The morphological closing
operation is analagous to the automatic behaviour of the human eye,
which is adept at smoothing over small-scale noise to identify a
single coherent structure.

\begin{figure}
\begin{center}
\includegraphics[width=16cm]{differences.eps}0
\caption{Plot (a) shows the running difference image between two
  consecutive persistence images. The wavefront is already
  evident. Plot (b) shows the image at the same time after applying
  the noise removal algorithm (Section \ref{sec:aware:cleaning}).
  Plot (c) shows the resulting image after the morphological closing
  operation is applied. Thas the effect of filling in small gaps in
  the detected wavefront \citep[e.g.][]{2002dip..book.....G}.}
\label{method_figure}
\end{center}
\end{figure}

The advantage of this approach is twofold. Firstly, we do not have to
fit a complex profile to noisy data in order to locate the
wavefront. Secondly, the RDPIs remove much more structure that is not
associated with a propagating bright front (Fig. 2, bottom row)
compared to the PBD images (Fig. 2, top-row), and therefore better
isolates the wavefront, making noise-cleaning easier (Fig. 4).  NEMO
\citep{2005SoPh..228..265P} uses integrals of annuli of RD images
(Fig. 2, middle row) to make their detections (Sec. 2.3). However, some
EUV waves do not propagate circularly (for example, wave A, Fig. 2)
and therefore the annular assumption can lead to a weakened detection.
Secondly, RD images contain dimming and brightening structure
unconnected with the EUV wavefront, and are noisier, when compared to
RDPIs, making isolation of the wavefront from these confounding
features more difficult.

The final result of the first part of AWARE is a time-series of images
that may contain an EUV wave.  The second part of AWARE searches for
the presence of a wave, assesses its quality, and characterizes its
propagation through the solar atmosphere.

\subsection{Determining wave presence and
  dynamics}\label{sec:aware:dynamics}

The output of the first part of AWARE is a time-series of images that
indicate regions that were progressively brightening.  The brightening
is assessed at the length-scales corresponding to the radii $r_{1},
r_{2}$\textellipsis\ of the median filter and morphological filters.
Hence the filtering operations have done the important job of at least
localizing where a wave might be.  The next step is to determine the
presence of a wave.

It is known that EUV waves are associated with solar eruptive events,
and so this provides a candidate location and start time for the
source of the wave. The candidate location is used as a pole to
transform the view of the Sun in AIA to a heliographic projection with
the EUV wave source location at one pole.  In this projection, waves
can be easily measured as they travel along lines of latitude
(constant longitude) in the new co-ordinate system.  These lines of
latitude are functionally equivalent to the lines of arc in
\citet{2014SoPh..289.3279L}, Figure 2(a, b).

At constant longitude, the position of the wavefront is calculated as
a function of time.  Therefore at each time, the position of the
wavefront and an estimate of the error in that position must be
calculated. \citet{2014SoPh..289.3279L} determine the location of the
wavefront by fitting a model of the intensity profile along the arc.
This is the same transform as implemented by
\citet{2014SoPh..289.3279L} and \citet{2005SoPh..228..265P}. 

%
% How to measure where the wavefront is
%

After the position and an error in the position of the wavefront have
been calculated as a function of time, the progress of the wave is fit
with a quadratic $s = vt + 0.5at^{2}$.  The fitting algorithm is 


features more difficult. The final result of the first component of
AWARE is a time-series of images that may contain an EUV wave.

The next step in the AWARE algorithm is to detect the presence of a
wave, assess its quality, and characterize its propagation through the
solar atmosphere. After the application of the image processing
methods described in Section \ref{img_proc}, the next step in the
detection and characterization process is to transform the solar image
so that the “north pole” of the new heliographic coordinate system is
the estimated origin of the EUV wave. In these coordinates, a wave
propagating uniformly across the Sun takes the form of an approximate
straight line, with its velocity entirely in a southward direction
[FIGURE EXAMPLE?].


The angular distance traversed at each image time is then estimated by
taking cross-sections or `slices' of the transformed images and
calculating the mean angular distance weighted by the intensity.
Error estimates are found by taking the standard deviation of the
angular distance weighted by the intensity. These angular distances
are fit with a quadratic equation to estimate the wave velocity and
acceleration.

A propagating wave is determined to be present if we can measure its
existence at at least one position along the wavefront, and at least
two times: this is the minimum amount of information required to
define a traveling feature.  CorPITA \citep{2014SoPh..289.3279L} use a
quality score to assess how well determined is each part of the EUV
wavefront. We adopt a similar apporach, with the intention of
extending its usage to make it a powerful and simple tool to let the
user decide what constitutes high and low quality EUV waves. For
example, by adopting several indicators of the quality of the
detection, the user may utilize the average and median scores to
determine the overall quality of the event detection.  The minimum,
maximum and standard deviation of the score lets the user determine
the spread in the detection quality.


\section{Results}\label{sec:results}

<<<<<<< HEAD
\begin{figure}
\begin{center}
\includegraphics[width=16cm]{euvwave_contour_map_corpita_fig7.eps}
\caption{Example detection}
\label{corpita_fig7}
\end{center}
\end{figure}

\begin{figure}
\begin{center}
\includegraphics[width=16cm]{euvwave_contour_map_corpita_fig8a.eps}
\caption{Example detection}
\label{corpita_fig8a}
\end{center}
\end{figure}

\begin{figure}
\begin{center}
\includegraphics[width=16cm]{euvwave_contour_map_corpita_fig6.eps}
\caption{Example detection}
\label{corpita_fig6}
\end{center}
\end{figure}

\begin{figure}
\begin{center}
\includegraphics[width=16cm]{euvwave_contour_map_corpita_fig8e.eps}
\caption{Example detection}
\label{corpita_fig8e}
\end{center}
\end{figure}

\begin{figure}
\begin{center}
\includegraphics[width=16cm]{euvwave_contour_map_previous1.eps}
\caption{Example detection}
\label{previous1}
\end{center}
\end{figure}
=======
\subsection{Detections with simulated wave data}

\subsection{Detections with SDO/AIA image data}

To illustrate the application of the AWARE algorithm to real solar imaging data, we examine a selection of recent events observed by SDO/AIA. These are the flares of 2011 February 13, 2011 February 15, 2011 February 16, and 2011 October 1. These flares serve as a convenient starting point for the search algorithm, given that EUV waves are considered to be triggered phenomenon. 

Figure \ref{results_figure} shows both the result of AWARE image processing (top panels), and wave characterization (bottom panels) steps. For all 4 events, the image processing reveals a propagating front of substantial extent. In Figure \ref{results_figure}, this is illustrated by the coloured maps in the top panels. These maps show the location of the brightest detected features as a function of time, ranging from blue (early times) to red (later times). Differences between the four events are immediately apparent; the 2011 February 15 event propagates outward in all directions approximately circularly. In contrast, the 2011 February 13 and 16 events propagate in narrow cones only, while the 2011 October 1 events extends in a significant arc towards solar southeast. 

For each event, we show an example of the wave characterization along a specific arc of interest (grey arcs, top panels). The arc selected for analysis in each case is the one with the highest quality score, as discussed in Section \ref{wave_char} The resulting velocity fits are shown in Figure \ref{results_figure} (bottom panels), where a quadratic profile has been fit to the angular distance of the detected wavefront. 

\begin{figure}
\begin{center}
\includegraphics[width=4cm]{euvwave_contour_map_previous1.eps}
\includegraphics[width=4cm]{euvwave_contour_map_corpita_fig7.eps}
\includegraphics[width=4cm]{euvwave_contour_map_corpita_fig8a.eps}
\includegraphics[width=4cm]{euvwave_contour_map_corpita_fig8e.eps}

\includegraphics[width=4cm]{previous1.dynamics.27.eps}
\includegraphics[width=4cm]{corpita_fig7.dynamics.2.eps}
\includegraphics[width=4cm]{corpita_fig8a.dynamics.40.eps}
\includegraphics[width=4cm]{corpita_fig8e.dynamics.71.eps}
\caption{The propagation of detected EUV waves as a function of time, for a) 2011 October 1, b) 2011 February 13, c) 2011 February 15, d) 2011 February 16. For each event, a background image of the Sun is shown, while the colours represent the location of the wavefront detected by AWARE at a given time.}
\end{center}
\end{figure}




>>>>>>> b85469fa74b83b4f42b1e586a6318236d8709bca


\input(5-Future_work}

\section{Conclusions and future work}\label{sec:conclusions}

Running difference persistence images (RDPIs) are a simple and novel
way of isolating faint, propagating bright features in coronal channel
AIA data.  The AWARE algorithm as presented uses these images as the
basis for detections and characterizations of EUV waves.  The
algorithm can be extended in various ways.  For example, AWARE uses
median filtering at a fixed length-scale to remove noise in the RDPIs.
A simple extension to this would be to apply the median filter
uniformly across the image, but use many different scales. Detections
from each scale-dependent median filtered image would be added up to
obtain a final composite which has removed noise at multiple
length-scales.  Another approach would be to use an adaptive median
filter.  Such median filters

As was noted in Section \ref{sec:results}, the AWARE detections shown in
Figures \ref{corpita_fig7} and \ref{corpita_fig8a} both show the
presence of the blooming of the CCD detector due to large flare
associated with these events.  These are clearly not part of the
wavefront.  In the future, the processing algorithm should be adjusted
to detect the presence of these features and mitigate their effects in
the subsequent detection.


A full AWARE algorithm would begin with a source event occuring at
some point in space and time.  AWARE would then automatically obtain
the data and perform the image processing and wave assessment steps,
finally writing a record into a database that is easily accessed
through existing solar feature and event clients.  AWARE does not yet
have that capability; however, the simple image processing algorithm
outlined here is the crucial part in turning the desire for automated
EUV wave event detection into a reliably functioning asset for the
solar physics community.


\section{Acknowledgements}
We are grateful to the developers of SSWIDL \citep{ssw}, IPython
\citep{ipython}, SunPy \citep{mumford-proc-scipy-2013}, PyMC
\citep{pymc2010}, matplotlib \citep{Hunter:2007:matplotlib} and the
Scientific Python stack for providing data preparation, manipulation,
analysis and display packages.  LH and JI acknowledge the support of
the Solar Data Analysis Center (SDAC).



\bibliography{eitwave-paper.bib}
\bibliographystyle(swsc}

\end{linenumbers}

\end{document}
