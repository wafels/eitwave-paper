\section{Conclusions and future work}\label{sec:conclusions}

Running difference persistence images (RDPIs) are a simple and novel
way of isolating faint, propagating bright features in coronal channel
AIA data.  The AWARE algorithm as presented uses these images as the
basis for detections and characterizations of EUV waves.  The
algorithm can be extended in various ways.  For example, AWARE uses
median filtering at a fixed length-scale to remove noise in the RDPIs.
A simple extension to this would be to apply the median filter
uniformly across the image, but use many different scales. Detections
from each scale-dependent median filtered image would be added up to
obtain a final composite which has removed noise at multiple
length-scales.  Another approach would be to use an adaptive median
filter.  Such median filters

As was noted in Section \ref{sec:results}, the AWARE detections shown in
Figures \ref{corpita_fig7} and \ref{corpita_fig8a} both show the
presence of the blooming of the CCD detector due to large flare
associated with these events.  These are clearly not part of the
wavefront.  In the future, the processing algorithm should be adjusted
to detect the presence of these features and mitigate their effects in
the subsequent detection.


A full AWARE algorithm would begin with a source event occuring at
some point in space and time.  AWARE would then automatically obtain
the data and perform the image processing and wave assessment steps,
finally writing a record into a database that is easily accessed
through existing solar feature and event clients.  AWARE does not yet
have that capability; however, the simple image processing algorithm
outlined here is the crucial part in turning the desire for automated
EUV wave event detection into a reliably functioning asset for the
solar physics community.
