\section{Conclusions and future work}\label{sec:conclusions}

Running difference persistence images (RDPIs) are a simple and novel
way of isolating faint, propagating bright features in coronal channel
AIA data.  The AWARE algorithm as presented uses these images as the
basis for detections and characterizations of EUV waves.  The
algorithm can be extended in various ways.  For example, the noise
removal step is intended to remove noise from the images, and leave
the wavefront isolated.  The method used in this paper sums the
results of the median filter applied at multiple length-scales.  But
there are a number of other algorithms that can also remove noise.
For example, \citet{2002dip..book.....G} describe an adaptive median
filter algorithm that varies the size-scale of the area over which
the median is calculated based on local greyscale values in the
image. Two-dimensional wavelets, along with thresholding algorithms
could be used to determine global (or local) noise levels, segmenting
the wavefront from the data.

An operational implementation of the AWARE algorithm would begin with
a source event occurring at some point in space and time.  Data
providers would be queried for the relevant image data.  Once that
data was available and downloaded, AWARE would be used to perform the
image processing and wave assessment steps, finally writing a record
into a database that is easily accessed through existing solar feature
and event databases.  AWARE does not yet have that capability;
however, the simple image processing algorithm outlined here is the
crucial part in turning the desire for automated EUV wave event
detection into a reliably functioning asset for the solar physics
community.

We have demonstrated the core functionality of AWARE, a new algorithm
for the detection and characterization of EUV waves in the solar
corona. As shown in Section \ref{sec:results}, AWARE performs
effectively, providing confirmed detections of wave both for simulated
wave data with known parameters, and also for a number of real EUV
wave events (see Figure \ref{results_figure}). In both cases, mimimal
human input was required in order to acheive the presented results.

The next steps in the development are already underway (see Section
\ref{sec:future}), including the construction of a consistent AWARE
data product for disemmination to the solar community. This product
will include metadata on any detected EUV waves, as well as image
masks of the detected events. Additional steps will include further
integration with online heliophysics tools, such as the HEK and VSO
services.
