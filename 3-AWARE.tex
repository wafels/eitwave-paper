\section{AWARE}\label{sec:aware}

The AWARE algorithm is in two parts.  The image processing part of
AWARE is intended to find locations where a propagating wavefront may
be located.  The 


\subsection{Image processing}
\label{img_proc}

As was noted in Sections 1 and 2, EUV waves are difficult to detect
since they are faint, extensive and propagate over a complex
background image (the solar corona). This realisation has driven past
attempts to enhance and detect EUV waves by making use of running
difference or base-difference images. However, these images, while
enhancing potential wavefronts, remain noisy and populated by other
extraneous features (see Figure \ref{rpdm_figure}). For AWARE, we
therefore choose a different approach, adopting a new, simple and very
promising strategy for segmenting an EUV wave wavefront from image
data. Instead of using traditional running- or base-difference image
processing, we make use of persistence imaging, as described by
\citet{2014AAS...22421838T}.

\subsection{Segmenting a wave from the image data}\label{sec:aware:segment}

A persistence image is found by calculating the persistence value of
the emission at each pixel at all locations and times.  The
persistence value at time $t$ of a time series $f(t)$ is simply the
maximum value reached by that pixel in the time range $0\rightarrow
t$.  If at later times the pixel value increases, the persistence
value increases accordingly. If the pixel value decreases however, the
persistence value remains unchanged. Hence, a set of persistence maps
constructed from an image sequence will indicate the brightest values
yet achieved in that sequence at each $t$.  The persistence transform
$P(t)$ of the time-series $f(t)$ is defined as
\begin{equation}
\label{eqn:persisttransform}
P(t) = \max_{t'=0}^{t'=t}f(t').
\end{equation}
Figure \ref{fig:persistence} shows the result of the pesistence
transform applied to some simulated data.  By obtaining sets of
persistence images from input solar EUV image data, and performing a
running difference operation on these images, rather than the original
input data, we are able to substantially enhance the appearance of
propagating waves. This is because running difference persistence
images (RDPIs) have two particularly desirable properties when
searching for EUV waves.  Firstly, only pixels that brighten over
previous values have a non-zero value in the running difference of
persistence images, while zero-value pixels correspond to areas that
did not increase in brightness. Hence, since an EUV wave brightens
neighboring pixels as it moves across the Sun, the RDPIs isolate those
brightening pixels.  In other words, the RDPIs isolate the leading
part of the wavefront that brightened new pixels.  Secondly, since
much of the corona does not vary significantly during an EUV wave,
RDPIs show very little residual coronal structure distant from the
EUV, greatly simplifying the resulting images.

\begin{figure}
\begin{center}
\includegraphics[width=16cm]{persistence_explanation.eps}
\caption{Example of the application of the persistence transform.  The
original data $f(t)$ is shown in blue, and its persistence transform
$P(t)$ is shown in green.}
\label{fig:persistence}
\end{center}
\end{figure}

\begin{figure}
\begin{center}
\includegraphics[width=16cm]{difference_comparison.eps}
\caption{Imaging processing techniques applied to three EUV waves,
  from 2011 October 1 (left column), 2011 February 13 (center column),
  and 2011 February 15 (right column) respectively. In all plots,
  black pixels indicate positive changes in emission, white indicates
  no change in emission, and red indicates a negative change in
  emission. The top row shows the running difference persistence
  images (RDPI), the middle row shows the running difference (RD)
  images (used in NEMO analysis, \citep{2005SoPh..228..265P}), and the
  bottom row shows the percentage base difference (PBD), used by the
  CorPITA algorithm \citep{2014SoPh..289.3279L}.  The RDPIs show a
  cleaner separation of the propagating bright wavefront compared to
  the other two differencing methods.  \cite{2014SoPh..289.3279L}; see
  their Fig. 7 and 8a respectively.}
\label{rpdm_figure}
\end{center}
\end{figure}

Figure \ref{rpdm_figure} illustrates the power of RDPIs for three
example EUV wave events from 2011 October 1 (Wave A), 2011 February 13
(Wave B), and 2011 February 15 (Wave C) respectively. For each event,
percentage base difference images (top row), running difference images
(center row) and RDPIs (bottom row) are shown. Waves B and C from
Figure \ref{rpdm_figure} were also analyzed by CorPITA in the
demonstration of that algorithm by \citet{2014SoPh..289.3279L}.  The
first row shows percentage base difference (PDB) images of each event,
the basic image type analyzed by the CorPITA algorithm.  The second
row shows running difference (RD) images, the basic image type
analyzed by the NEMO algorithm.  The third row shows RDPI images, the
basic image type analyzed by AWARE. Comparison with RDPIs shows that
in standard RD or PBD images the wavefront is more diffuse, and much
coronal structure not associated with the wavefront remains in the
image. RD and PBD images also have much denser noise compared to the
RDPIs of the same data; hence, separating the EUV wave from the noise
is substantially easier when using RDPIs.

RPD images are the first step in the detection of EUV waves with
AWARE. Subsequent image processing steps allow us to refine the
detection of any propagating features. The major steps in the AWARE
detection and processing algorithm are described below, and are
demonstrated in Figure \ref{method_figure}.

\begin{enumerate}

\item Given a set of sequential solar EUV images (e.g. SDO/AIA or
  STEREO/EUVI data), data are summed in time and space to increase the
  signal to noise ratio of the wave against the background. Images may
  be summed in space as desired, for example an AIA image may be
  binned using $4\times4$ super-pixels to form $1024\times1024$ pixel
  images. In the time dimension, images may be summed as
  required. Typically, pairs or triplets of consecutive images are
  used.

\item The persistence transform is then applied to the resulting image
  set.  This creates a set of persistence images, showing the
  brightest values obtained in each pixel (see Equation
  \ref{eqn:persisttransform}) as a function of time.

\item Perform a running difference operation on the
  persistence images. Hence, only areas that increase in brightness
  from one time to the next remain (Figure \ref{method_figure}b).

\item 
The following steps are intended to isolate the wavefront at different
length-scales $r_{1}, r_{2}$\textellipsis, etc, from the noise in the
datacube.

\begin{enumerate}

\item All pixel values in the data cube above a certain threshold are
  set to zero. This filters out spikes in the data.  The data is then
  rescaled to the range zero to 1.

\item Apply a noise reduction filter (Figure \ref{method_figure}c) to
  the data cube of images.  Our demonstration algorithm uses a simple
  median filter.  This replaces every pixel in the image with the
  median value found in its neighborhood.  The median filter used is a
  circle in the spatial dimensions with a given radius $r$.  The
  median filter also extends in the temporal dimension to the previous
  and next images in the data cube.  This extension in the temporal
  dimension improves the noise reduction since emission at recent
  times are compared.  The median filter is a commonly used and simple
  method of removing noise from an image \citep{2002dip..book.....G}.

\item Apply a morphological closing operation.  This operation helps
  close small ‘cracks’ in structures \citep{2002dip..book.....G}.  The
  structuring element used is the same as that used by the median
  filtering operation. The final image is shown in Figure
  \ref{method_figure}d.  The final result is a datacube of images that
  show the location of the wavefront as a function of time, given the
  median filter and closing operations performed at length-scale $r$.
\end{enumerate}

\item Add up the wave location datacubes over all the circle radii
  $r_{1}, r_{2}$\textellipsis, etc.  This is the final wave location
  datacube that defines the location of the wavefront as a function of
  time.

\end{enumerate}

The final product is a datacube of time-ordered series of images that
localize the bright wavefront of the EUV wave.  Figure
\ref{method_figure} shows each step in this procedure, applied to two
AIA 211 $\AA$ images during the EUV wave event of 2011 October 1 (Wave
A in Figure \ref{rpdm_figure}).  The result of the key step of
generating an RDPI (Steps 2 and 3) from the two input images is
illustrated in panel b). Steps 4 and 5 of the AWARE image processing
method (Figure \ref{method_figure}c, d) apply some basic noise
reduction and feature enhancement techniques with the intent to
further isolate the wave front in the data. The morphological closing
operation is analagous to the automatic behaviour of the human eye,
which is adept at smoothing over small-scale noise to identify a
single coherent structure.

\begin{figure}
\begin{center}
\includegraphics[width=16cm]{show_aware_processing.eps}
\caption{Plot (a) shows the running difference image between two
  consecutive persistence images. The wavefront is already
  evident. Plot (b) shows the image at the same time after applying
  the noise removal algorithm (Section \ref{sec:aware:cleaning}).
  Plot (c) shows the resulting image after the morphological closing
  operation is applied. Thas the effect of filling in small gaps in
  the detected wavefront \citep[e.g.][]{2002dip..book.....G}.}
\label{method_figure}
\end{center}
\end{figure}

The advantage of this approach is twofold. Firstly, we do not have to
fit a complex profile to noisy data in order to locate the
wavefront. Secondly, the RDPIs remove much more structure that is not
associated with a propagating bright front (Fig. 2, bottom row)
compared to the PBD images (Fig. 2, top-row), and therefore better
isolates the wavefront, making noise-cleaning easier (Fig. 4).  NEMO
\citep{2005SoPh..228..265P} uses integrals of annuli of RD images
(Fig. 2, middle row) to make their detections (Sec. 2.3). However, some
EUV waves do not propagate circularly (for example, wave A, Fig. 2)
and therefore the annular assumption can lead to a weakened detection.
Secondly, RD images contain dimming and brightening structure
unconnected with the EUV wavefront, and are noisier, when compared to
RDPIs, making isolation of the wavefront from these confounding
features more difficult.

The final result of the first part of AWARE is a time-series of images
that may contain an EUV wave.  The second part of AWARE searches for
the presence of a wave, assesses its quality, and characterizes its
propagation through the solar atmosphere.

\subsection{Determining wave presence and
  dynamics}\label{sec:aware:dynamics}

The output of the first part of AWARE is a time-series of images that
indicate regions that were progressively brightening.  The brightening
is assessed at the length-scales corresponding to the radii $r_{1},
r_{2}$\textellipsis\ of the median filter and morphological filters.
Hence the filtering operations have done the important job of at least
localizing where a wave might be.  The next step is to determine the
presence of a wave.

It is known that EUV waves are associated with solar eruptive events,
and so this provides a candidate location and start time for the
source of the wave. The candidate location is used as a pole to
transform the view of the Sun in AIA to a heliographic projection with
the EUV wave source location at one pole.  In this projection, waves
can be easily measured as they travel along lines of latitude
(constant longitude) in the new co-ordinate system.  These lines of
latitude are functionally equivalent to the lines of arc in
\citet{2014SoPh..289.3279L}, Figure 2(a, b).

At constant longitude, the position of the wavefront is calculated as
a function of time.  Therefore at each time, the position of the
wavefront and an estimate of the error in that position must be
calculated. \citet{2014SoPh..289.3279L} determine the location of the
wavefront by fitting a model of the intensity profile along the arc.
This is the same transform as implemented by
\citet{2014SoPh..289.3279L} and \citet{2005SoPh..228..265P}. 

%
% How to measure where the wavefront is
%

After the position and an error in the position of the wavefront have
been calculated as a function of time, the progress of the wave is fit
with a quadratic $s = vt + 0.5at^{2}$.  The fitting algorithm is 


features more difficult. The final result of the first component of
AWARE is a time-series of images that may contain an EUV wave.

The next step in the AWARE algorithm is to detect the presence of a
wave, assess its quality, and characterize its propagation through the
solar atmosphere. After the application of the image processing
methods described in Section \ref{img_proc}, the next step in the
detection and characterization process is to transform the solar image
so that the “north pole” of the new heliographic coordinate system is
the estimated origin of the EUV wave. In these coordinates, a wave
propagating uniformly across the Sun takes the form of an approximate
straight line, with its velocity entirely in a southward direction
[FIGURE EXAMPLE?].


The angular distance traversed at each image time is then estimated by
taking cross-sections or `slices' of the transformed images and
calculating the mean angular distance weighted by the intensity.
Error estimates are found by taking the standard deviation of the
angular distance weighted by the intensity. These angular distances
are fit with a quadratic equation to estimate the wave velocity and
acceleration.

A propagating wave is determined to be present if we can measure its
existence at at least one position along the wavefront, and at least
two times: this is the minimum amount of information required to
define a traveling feature.  CorPITA \citep{2014SoPh..289.3279L} use a
quality score to assess how well determined is each part of the EUV
wavefront. We adopt a similar apporach, with the intention of
extending its usage to make it a powerful and simple tool to let the
user decide what constitutes high and low quality EUV waves. For
example, by adopting several indicators of the quality of the
detection, the user may utilize the average and median scores to
determine the overall quality of the event detection.  The minimum,
maximum and standard deviation of the score lets the user determine
the spread in the detection quality.
